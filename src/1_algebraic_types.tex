\section{Algebraic Types}
  \subsection{Types 1-3: $m\degree$ Polynomial over $n\degree$ Polynomial}
    \subsubsection{Necessary formulae for types 1-3}
      \begin{enumerate}
        \item $\int \frac{\ \mathrm{d}x}{x^2 + a^2} = \frac{1}{a} \arctan(\frac{x}{a}) + c$
        \item $\int \frac{\ \mathrm{d}x}{x^2 - a^2}= \frac{1}{2a} \ln |\frac{x-a}{x+a}| + c$
        \item $\int \frac{\ \mathrm{d}x}{a^2 - x^2} = \frac{1}{2a} \ln |\frac{a+x}{a-x}| + c$
        \item $\int \frac{f'(x)}{f(x)} \ \mathrm{d}x = \ln |f(x)| + c$
      \end{enumerate}

    \subsubsection{Type 1: $0\degree$ Polynomial over $2\degree$ Polynomial}
      
      \begin{equation*}
        \int \frac{M}{Ax^2 + Bx + C} \ \mathrm{d}x
      \end{equation*}
      
      \begin{center}
        \textbf{Rule:} We have to modify the denominator as follows: \begin{align*}
          & \int \frac{M}{Ax^2 + Bx + C} \ \mathrm{d}x \\
          & = \frac{M}{A} \int \frac{1}{x^2 + \frac{B}{A} x + \frac{C}{A}} \ \mathrm{d}x
          = \frac{M}{A} \int \frac{\ \mathrm{d}x}{(x \pm \alpha)^2 \pm \beta^2}
        \end{align*}
      \end{center}
      
      \textbf{Examples:} \\
        
        \begin{enumerate}
          
          \item \begin{flalign*}
            \int \frac{3\ \mathrm{d}x}{2x^2 + 4x + 20}
              & = \frac{3}{2} \int \frac{\ \mathrm{d}x}{x^2 + 2x + 10} &\\
              & = \frac{3}{2} \int \frac{\ \mathrm{d}x}{(x^2 + 2x + 1) + 9} \\
              & = \frac{3}{2} \int \frac{\ \mathrm{d}x}{(x+1)^2 + 3^2} \\
              & = \frac{\cancel{3}}{2} \times \frac{1}{\cancel{3}} \arctan (\frac{x+1}{3}) + c \\
              & = \frac{1}{2} \arctan (\frac{x+1}{3}) + c
            \end{flalign*}
          
          \item \begin{flalign*}
            \int \frac{\ \mathrm{d}x}{x^2 + 2x - 3}
              & = \int \frac{\ \mathrm{d}x}{(x^2 + 2x + 1) - 4} & \\
              & = \frac{1}{2\times2} \ln |\frac{x+1-2}{x+1+2}| + c \\
              & = \frac{1}{4} \ln |\frac{x-1}{x+3}| + c
          \end{flalign*}
        
        \end{enumerate}
    
     \subsubsection{Type 2: $1\degree$ Polynomial over $2\degree$ Polynomial}
      
      \begin{equation*}
        \int \frac{Mx + N}{Ax^2 + Bx + C} \ \mathrm{d}x
      \end{equation*}
      
      \begin{center}
        \textbf{Rule:} We have to write the nominator in the format $P \frac{\mathrm{d}}{\mathrm{d}x}\mathrm{(denominator)} + Q$, and use Type 1 as needed. For example:

        \begin{align*}
          & \int \frac{Mx + N}{Ax^2 + Bx + C} \ \mathrm{d}x \\
          & = \int \frac{P(2Ax + B) + Q}{Ax^2 + Bx + C} \ \mathrm{d}x \\
          & = P \int \frac{2Ax + B}{Ax^2 + Bx + C} \ \mathrm{d}x + \frac{Q}{A} \int \frac{1}{x^2 + \frac{B}{A} x + \frac{C}{A}}\ \mathrm{d}x \\
          & = P \int \frac{2Ax + B}{Ax^2 + Bx + C} \ \mathrm{d}x + \frac{Q}{A} \int \frac{\ \mathrm{d}x}{(x \pm \alpha)^2 \pm \beta^2}
        \end{align*}

      \end{center}
      
      \textbf{Examples:} \\
        
        \begin{enumerate}
        
          \item \begin{flalign*}
              \int \frac{3x + 5}{x^2 + 4x - 5} dx
            & = \int \frac{\frac{3}{2} (2x + 4) - 1}{x^2 + 4x - 5} dx & \\
            & = \frac{3}{2} \int \frac{2x + 4}{x^2 + 4x - 5}\ \mathrm{d}x - \int \frac{1}{x^2 + 4x - 5}dx\\
            & = \frac{3}{2} \ln |x^2 + 4x - 5| - \int \frac{1}{(x + 2)^2 - 3^2}dx \\
            & = \frac{3}{2} \ln |x^2 + 4x - 5| - \frac{1}{6} \ln |\frac{x - 1}{x + 5}| + c\\
            \end{flalign*}
          
          \item \begin{flalign*}
              \int \frac{5x + 7}{4x^2 + 32x - 100} dx
            & = \int \frac{\frac{5}{8} (8x + 32) - 13}{4x^2 + 32x - 100} dx & \\
            & = \frac{5}{8} \int \frac{8x + 32}{4x^2 + 32x - 100}\ \mathrm{d}x - \int \frac{13}{4x^2 + 32x - 100}dx\\
            & = \frac{5}{8} \ln |4x^2 + 32x - 100| - \frac{13}{4} \int \frac{1}{x^2 + 8x - 25}dx + c\\
            & = \frac{5}{8} \ln |x^2 + 8x - 25| - \frac{13}{4} \int \frac{1}{(x + 4)^2 - (\sqrt{41})^2}dx + c\\
            & = \frac{5}{8} \ln |x^2 + 8x - 25| - \frac{13}{8 \sqrt{41}} \ln |\frac{x + 4 - \sqrt{41}}{x+4+\sqrt{41}}| + c\\
            \end{flalign*}
        
        \end{enumerate} 
    
     \subsubsection{Type 3: $2\degree$ Polynomial over $2\degree$ Polynomial}
      
      \begin{equation*}
        \int \frac{Mx^2 + Nx + L}{Ax^2 + Bx + C} \ \mathrm{d}x
      \end{equation*}
      
      \begin{center}
        \textbf{Rule:} We have to write the nominator in the format $P\mathrm{(denominator)} + Q \frac{\mathrm{d}}{\mathrm{d}x}\mathrm{(denominator)} + R$, and use Types 1 \& 2 as needed. For example:

        \begin{align*}
          & \int \frac{Mx^2 + Nx + L}{Ax^2 + Bx + C} \ \mathrm{d}x \\
          & = \int \frac{P(Ax^2 + Bx + C) + Q(2Ax + B) + R}{Ax^2 + Bx + C} \ \mathrm{d}x \\
          & = P\int\mathrm{d}x +   Q \int\frac{2Ax + B}{Ax^2 + Bx + C} \ \mathrm{d}x + \frac{R}{A} \int \frac{1}{x^2 + \frac{B}{A} x + \frac{C}{A}}\ \mathrm{d}x \\
          & = Px + Q \ln |Ax^2 + Bx + C| + \frac{Q}{A} \int \frac{\ \mathrm{d}x}{(x \pm \alpha)^2 \pm \beta^2}
        \end{align*}

      \end{center}

      \textbf{Examples:}
        
        \begin{enumerate}

          \item \begin{flalign*}
              \int \frac{2x^2 + 3x + 1}{x^2 + 4x + 13}\mathrm{d}x
              & = \int \frac{2(x^2 + 4x + 13) - \frac{5}{2}(2x+4) - 15}{x^2 + 4x + 13}\mathrm{d}x & \\
              & = \int 2\mathrm{d}x -\frac{5}{2} \int \frac{2x+4}{x^2 + 4x + 13}\mathrm{d}x - 15 \int \frac{\mathrm{d}x}{x^2 + 4x + 13} \\
              & = 2x - \frac{5}{2} \ln |x^2 + 4x + 13| - 15 \int \frac{\mathrm{d}x}{(x+2)^2 + 3^2} \\ 
              & = 2x - \frac{5}{2} \ln |x^2 + 4x + 13| - \ ^{5}\cancel{15} \times \frac{1}{\cancel{3}} \arctan(\frac{x+2}{3}) + c \\ 
              & = 2x - \frac{5}{2} \ln |x^2 + 4x + 13| - 5 \arctan(\frac{x+2}{3}) + c \\ 
          \end{flalign*}
          
          \item \begin{flalign*}
              \int \frac{3x^2 + 7x + 10}{x^2 - 6x + 25} \mathrm{d}x
              & = \int \frac{3(x^2 - 6x + 25) + \frac{25}{2} (2x-6) + 10}{x^2 - 6x + 25} \mathrm{d}x & \\
              & = 3x + \frac{25}{2}\ln|x^2 - 6x + 25| + 10 \int \frac{\mathrm{d}x}{(x-3)^2 + 4^2} \\
              & = 3x + \frac{25}{2}\ln|x^2 - 6x + 25| + \ ^{5}\cancel{10} \frac{1}{\cancel{4}_{2}} \arctan(\frac{x-3}{4}) + c \\
              & = 3x + \frac{25}{2}\ln|x^2 - 6x + 25| + \frac{5}{2} \arctan(\frac{x-3}{4}) + c 
          \end{flalign*}

        \end{enumerate}

      \paragraph{Bonus: $(>m\degree)$ Polynomial over $m\degree$ Polynomial}
        \begin{center}
          \textbf{Rule:} Use the remainder theorem to split the nominator into more manageable parts.
        \end{center}

        \textbf{Examples: }
          
          \begin{enumerate}
            
            \item \begin{flalign*}
                \int \frac{x^6}{x^2 + 1} \mathrm{d}x
                & = \int \frac{x^4(x^2 + 1) - x^2(x^2 + 1) + 1(x^2 + 1) - 1}{x^2 + 1} \mathrm{d}x & \\
                & = \int (x^4 - x^2 +1) \mathrm{d}x - \int \frac{\mathrm{d}x}{x^2 + 1} \\
                & = \frac{x^5}{5} - \frac{x^3}{3} + x - \arctan(x) + c \\
            \end{flalign*}

            \item \begin{flalign*}
              \int \frac{5x^3 + 7x^2 + 2x + 1}{x^2 - 4x + 13} \mathrm{d}x
              & = \int \frac{5x(x^2 - 4x + 13) + (27x^2 - 63x + 1}{x^2 - 4x + 13} & \\
          & = \int 5x \mathrm{d}x + \int \frac{27x^2 - 63x + 1}{x^2 - 4x + 13} \mathrm{d}x \\         
            \end{flalign*}

          \end{enumerate}


  \subsection{Types 4-6: $m\degree$ Polynomial over $\sqrt{n\degree}$ Polynomial}

    \subsubsection{Necessary formulae for types 4-6}
      
      \begin{enumerate}
        \item $\int \frac{1}{\sqrt{x^2 + a^2}}\ \mathrm{d}x = \ln |\sqrt{x^2 + a^2} + x| + c$
        \item $\int \frac{1}{\sqrt{x^2 - a^2}}\ \mathrm{d}x = \ln |\sqrt{x^2 - a^2} + x| + c$
        \item $\int \frac{1}{\sqrt{a^2 - x^2}}\ \mathrm{d}x = \arcsin(\frac{x}{a}) + c$
        \item $\int \sqrt{x^{2} + a^{2}}\ \mathrm{d}x = \frac{x}{2} \sqrt{x^{2} + a^{2}} + \frac{a^{2}}{2} \ln |\sqrt{x^{2} + a^{2}} + x| + c$
        \item $\int \sqrt{x^{2} - a^{2}}\ \mathrm{d}x = \frac{x}{2} \sqrt{x^{2} - a^{2}} - \frac{a^{2}}{2} \ln |\sqrt{x^{2} - a^{2}} + x| + c$
        \item $\int \sqrt{a^{2} - x^{2}}\ \mathrm{d}x = \frac{x}{2} \sqrt{a^{2} - x^{2}} + \frac{a^{2}}{2} \arcsin(\frac{x}{a}) + c$
      \end{enumerate}

    \subsubsection{Type 4: $0\degree$ Polynomial over $\sqrt{2\degree}$ Polynomial}
      
      \begin{equation*}
        \int \frac{M}{\sqrt{Ax^2 + Bx + C}} \ \mathrm{d}x
      \end{equation*}
      
      \begin{center}
        \textbf{Rule:} Restructure the denominator as sum of or difference between two squares, same as type 1.
      \end{center}

      \textbf{Examples:}
        
        \begin{flalign*}
          \int \frac{\mathrm{d}x}{\sqrt{2x^2 + 2x + 3}}
          & = \frac{1}{\sqrt{2}} \int \frac{\mathrm{d}x}{\sqrt{(x + \frac{1}{2})^2 + (\frac{\sqrt{3}}{2})^2}} & \\
          & = \frac{1}{\sqrt{2}} \ln |\sqrt{x^2 + x + \frac{3}{2}} + x + \frac{1}{2}| + c 
        \end{flalign*}

    \subsubsection{Type 5: $1\degree$ Polynomial over $\sqrt{2\degree}$ Polynomial}

      \begin{equation}
        \int \frac{Mx + N}{\sqrt{Ax^{2} + Bx + C}} \mathrm{d}x
      \end{equation}

      \begin{center}
        \textbf{Rule:} Restructure the nominator as $P \frac{\mathrm{d}}{\mathrm{d}x} (denominator)^{2} + Q$ and apply type 4 as necessary.
      \end{center}

      \textbf{Examples:}

      \begin{flalign*}
        \int \frac{2x + 3}{\sqrt{3x^{2} - 12x + 27}} \mathrm{d}x
        & = \frac{1}{\sqrt{3}} \int \frac{2x + 3}{\sqrt{x^{2} - 4x + 9}} \mathrm{d}x & \\
        & = \frac{1}{\sqrt{3}} \int \frac{2x-4}{\sqrt{x^{2} - 4x + 9}} \mathrm{d}x + \frac{7}{\sqrt{3}} \int \frac{\mathrm{d}x}{\sqrt{(x - 2)^{2} + (\sqrt{3})^{2}}} \\
        & = \frac{2}{\sqrt{3}} \sqrt{x^{2} - 4x + 9} + \frac{7}{\sqrt{3}} \ln |\sqrt{x^{2} - 4x + 9} + x - 2| + c
      \end{flalign*}


    \subsubsection{Type 6: $2\degree$ Polynomial over $\sqrt{2\degree}$ Polynomial}

      \begin{equation}
        \int \frac{Mx^{2} + Nx + L}{\sqrt{Ax^{2} + Bx + C}} \mathrm{d}x
      \end{equation}

      \begin{center}
        \textbf{Rule:} Restructure the nominator as $P (denominator)^{2} + Q \frac{\mathrm{d}}{\mathrm{d}x} (denominator)^{2} + R$ and apply types 4 and 5 as necessary.
      \end{center}

      \textbf{Examples:}

      \begin{flalign*}
        \int \frac{3x^{2} + 7x + 2}{\sqrt{x^{2} + 4x + 13}} \mathrm{d}x
        & = 3 \int \sqrt{x^{2} + 4x + 13}\ \mathrm{d}x - \frac{5}{2} \int \frac{2x + 4}{\sqrt{x^{2} + 4x + 13}} \mathrm{d}x - 27 \int \frac{\mathrm{d}x}{}\sqrt{x^{2} + 4x + 13} & \\
        & = 3 \int \sqrt{(x + 2)^{2} + 3^{2}}\ \mathrm{d}x - \frac{5}{2} \times 2 \sqrt{x^{2} + 4x + 13} - 27 \int \frac{\mathrm{d}x}{\sqrt{(x + 2)^{2} + 3^{2}}} \\
        & = 3 \left[ \frac{x + 2}{2} \sqrt{x^{2} + 4x + 13} + \frac{3^{2}}{2} \ln |\sqrt{x^{2} + 4x + 13} + x + 2| \right] \\
        & \quad  - 5 \sqrt{x^{2} + 4x + 13} - 27 \ln |\sqrt{x^{2} + 4x + 13} + x + 2| + c \\
        & = \frac{3(x + 2)}{2} \sqrt{x^{2} + 4x + 13} + \frac{27}{2} \ln |\sqrt{x^{2} + 4x + 13} + x + 2| \\
        & \quad  - 5 \sqrt{x^{2} + 4x + 13} - 27 \ln |\sqrt{x^{2} + 4x + 13} + x + 2| + c \\
        & = \frac{3x - 4}{2} \sqrt{x^{2} + 4x + 13} - \frac{27}{2} \ln |\sqrt{x^{2} + 4x + 13} + x + 2| + c
      \end{flalign*}

    \subsection{Type 7: $1\degree$ Polynomial over ${1\degree}$ Polynomial}

      \begin{equation}
        \int \frac{ax + b}{cx + f} \mathrm{d}x
      \end{equation}

      \begin{center}
        \textbf{Rule:} Let $cx + f$ be $z$, and apply basic formulas as necessary
      \end{center}

      \textbf{Examples:}

      \begin{align*}
        & \int \frac{4x + 3}{2x - 1} \mathrm{d}x & let\ 2x - 1 = z\ &\ \implies 2x = z + 1\\
        & = \frac{1}{2} \int \frac{2z + 5}{z} \mathrm{d}z & \mathrm{d}x = \frac{1}{2} \mathrm{d}z\ &\ \implies 4x + 3= 2z + 5 \\
        & = \frac{1}{2} \int 2 \mathrm{d}z + \frac{5}{2} \int \frac{\mathrm{d}z}{z} &&\\
        & = z + \frac{5}{2} \ln{|z|} + c &&\\
        & = 2x + \frac{5}{2} \ln{|2x - 1|} + c
      \end{align*}

    \subsection{Type 8: $1\degree$ Polynomial multiplied or divided by $\sqrt{1\degree}$ Polynomial}

      \begin{equation}
        \int \frac{ax + b}{\sqrt{cx + f}} \mathrm{d}x \ or, \ \int (ax + b)\sqrt{cx + f} \mathrm{d}x \ or, \ \int \frac{\mathrm{d}x}{(ax + b)\sqrt{cx + f}}
      \end{equation}

      \begin{center}
        \textbf{Rule:} Let $cx + f$ be $z^{2}$, and apply type 4 and other basic formulas as necessary
      \end{center}

      \textbf{Examples:}

      \begin{align*}
        & \int \frac{4x + 3}{\sqrt{2x - 1}} \mathrm{d}x & let\ 2x - 1 = z^{2}\ &\\
        & = \int \frac{2z^{2} + 5}{\cancel{z}} \cancel{z}\mathrm{d}z & 2\mathrm{d}x = 2z\mathrm{d}z\ &\ \implies \mathrm{d}x = z\mathrm{d}z \\
        & = \int 2z^{2} + 5 \mathrm{d}z &&\\
        & = \frac{2}{3} z^{3} + 5z + c && \\
        & = \frac{2}{3} (2x - 1)^{3} + 10x + c &&
      \end{align*}

    \subsection{Type 9: $\sqrt{1\degree}$ Polynomial over $\sqrt{1\degree}$ Polynomial, with nominator and denominator being conjugate}

      \begin{equation}
        \int \sqrt{\frac{a \pm x}{a \mp x}} \mathrm{d}x
      \end{equation}

      \begin{center}
        \textbf{Rule:} Multiply both nominators and denominators by denominator and apply type 4 as necessary.
      \end{center}

      \textbf{Examples:}

      \begin{flalign*}
        \int \sqrt{\frac{5 - x}{5 + x}} \mathrm{d}x
        & = \int \frac{5 - x}{\sqrt{25 - x^{2}}} \mathrm{d}x &\\
        & = \int \frac{\frac{1}{2}(-2x) + 5}{\sqrt{25 - x^{2}}} \\
        & = \sqrt{25 - x^{2}} + 5\sin^{-1}{\frac{x}{5}} + c
      \end{flalign*}
    
    \subsection{Types 10-15: Inverse of Polynomials}

      \subsubsection{Type 10: Inverse of $1\degree$ Polynomial multiplied by $\sqrt{2\degree}$ Polynomial}
      
        \begin{equation}
          \int \frac{\mathrm{d}x}{(ax+b)\sqrt{cx^2 + ex + f}}
        \end{equation}

        \begin{center}
          \textbf{Rule:} Let $ax+b=\frac{1}{z}$, and apply other rules if necessary.
        \end{center}

        \textbf{Examples:}

        \begin{flalign*}
          & \int \frac{\mathrm{d}x}{(x+1)\sqrt{3x^2+2x+1}} & let \ x+1=\frac{1}{z} \ \therefore z=\frac{1}{x+1} &\\
          & = \int \frac{-\frac{1}{z^2}\mathrm{d}z}{\frac{1}{z}\sqrt{3(\frac{1}{z}-1)^2 + 2(\frac{1}{z}-1)+1}} & x=\frac{1}{z}-1 &\ \implies \mathrm{d}z=-\frac{1}{x^2}\mathrm{d}x \\
          & = \int \frac{-\frac{1}{z^2}\mathrm{d}z}{\frac{1}{z}\times\frac{1}{z} \sqrt{3-4z+2z^2}} \\
          & = - \int \frac{\mathrm{d}z}{\sqrt{2z^2-4z+3}} \\
          & = -\frac{1}{\sqrt{2}} \int \frac{\mathrm{d}z}{\sqrt{z^2-2z+\frac{3}{2}}} \\
          & = -\frac{1}{\sqrt{2}} \int \frac{\mathrm{d}z}{\sqrt{(z-1)^2 + (\frac{1}{\sqrt{2}})^2}} \\
          & = -\frac{1}{\sqrt{2}} \ln |\sqrt{z^2 - 2z + \frac{3}{2}} + z - 1| + c;\ where\ z = \frac{1}{x+1}
        \end{flalign*}

      \subsubsection{Type 11: Inverse of $1\degree$ Polynomial multiplied by $n\degree$ Polynomial}

        \begin{equation}
          \int \frac{\mathrm{d}x}{x(a+bx^n)}
        \end{equation}

        \begin{center}
          \textbf{Rule:} Let $x^n = \frac{1}{z}$ ($x^n = z$ will also work, but can be tedious), take $\ln$, and apply other rules if necessary.
        \end{center}

        \textbf{Examples:}

        \begin{flalign*}
          & \int \frac{\mathrm{d}x}{x(4+5x^{20})} & let \ x^{20}=\frac{1}{z} \ \implies 20\ln{x} = -\ln{z} &\\
          & = -\frac{1}{20} \int \frac{\mathrm{d}z}{z(4+\frac{5}{z})} & \implies 20\frac{\mathrm{d}x}{x} = -\frac{\mathrm{d}z}{z} \\
          & = -\frac{1}{20} \int \frac{\mathrm{d}z}{4z+5} \\
          & = -\frac{1}{80}\ln{|4z+5|} + c \\
          & = -\frac{1}{80}\ln{|\frac{4}{x^{20}}+5|} + c
        \end{flalign*}

      \subsubsection{Type 12: Inverse of $1\degree$ Polynomial multiplied by $\sqrt{n\degree}$ Polynomial}

        \begin{equation}
          \int \frac{\mathrm{d}x}{x\sqrt{a+bx^n}}
        \end{equation}

        \begin{center}
          \textbf{Rule:} Let $x^n = \frac{1}{z^2}$, take $\ln$, and apply other rules if necessary.
        \end{center}

        \textbf{Examples:}

        \begin{flalign*}
          & \int \frac{\mathrm{d}x}{x\sqrt{5+x^{\frac{1}{3}}}} & let \ x^{\frac{1}{3}}=\frac{1}{z^2} \ \implies \frac{1}{3}\ln{x} = -2\ln{z} &\\
          & = -6\int \frac{\mathrm{d}z}{z\sqrt{5+\frac{1}{z^2}}} & \implies \frac{1}{3}\frac{\mathrm{d}x}{x} = -2\frac{\mathrm{d}z}{z} \\
          & = -6 \int \frac{\mathrm{d}z}{\sqrt{5z^2+1}} \\
          & = -\frac{6}{\sqrt{5}} \int \frac{\mathrm{d}z}{\sqrt{z^2+\frac{1}{5}}} \\
          & = -\frac{6}{\sqrt{5}} \ln|\sqrt{z^2+\frac{1}{5}} + z| + c \\
          & = -\frac{6}{\sqrt{5}} \ln|\sqrt{\frac{1}{x^{\frac{1}{3}}}+\frac{1}{5}} + \frac{1}{x^{\frac{1}{6}}}| + c
        \end{flalign*}

      \subsubsection{Type 13: Inverse of the sum or difference of two $\sqrt{1\degree}$ Polynomials}

        \begin{equation}
          \int \frac{\mathrm{d}x}{\sqrt{ax+b} \pm \sqrt{ax+c}}
        \end{equation}

        \begin{center}
          \textbf{Rule:} Multiply both nominator and denominator by the denominator's conjugate, and then use other rules as necessary.
        \end{center}

        \textbf{Examples:}

        \begin{flalign*}
          & \int \frac{\mathrm{d}x}{\sqrt{2x+5} - \sqrt{2x+3}} & \\
          & = \int \frac{\sqrt{2x+5} + \sqrt{2x+3}}{2x+5-2x-3}\  \mathrm{d}x \\
          & = \frac{1}{2} \left(\int\sqrt{2x+5}\ \mathrm{d}x + \int\sqrt{2x+3}\ \mathrm{d}x\right) \\
          & = \frac{1}{2}\times\frac{1}{2}\times\frac{2}{3}\left\{ \sqrt{(2x+5)^3} + \sqrt{(2x+3)^3} \right\} + c\\
          & = \frac{1}{6} \left\{ \sqrt{(2x+5)^3} + \sqrt{(2x+3)^3} \right\} + c
        \end{flalign*}
        \\ \\
      \subsubsection{Type 14: Inverse of product of two $\sqrt{1\degree}$ Polynomials}

        \begin{equation}
          \int \frac{\mathrm{d}x}{\sqrt{(x+\alpha)(x+\beta)}}
        \end{equation}

        \begin{center}
          \textbf{Rule:} Let $\sqrt{x+\alpha} + \sqrt{x+\beta} = z$, and then use other rules as necessary.
        \end{center}

        \textbf{Examples:}

        \begin{flalign*}
          & \int \frac{\mathrm{d}x}{\sqrt{(x-2)(x-3)}} & let\ z=\sqrt{x-2} + \sqrt{x-3} & \\
          & = \int \frac{2\mathrm{d}z}{z} & \implies \mathrm{d}z = \frac{1}{2\sqrt{x-2}} + \frac{1}{2\sqrt{x-3}} \\
          & = 2\ln|\sqrt{x-2} + \sqrt{x-3}| + c & \implies 2\mathrm{d}z = \frac{\sqrt{x-2}+\sqrt{x-3}}{\sqrt{(x-2)(x-3)}}\mathrm{d}x \\
          & & \implies \frac{2\mathrm{d}z}{z} = \frac{\mathrm{d}x}{\sqrt{(x-2)(x-3)}}
        \end{flalign*}

        \begin{center}
          \textbf{Note:} For MCQ, \[
            \int \frac{\mathrm{d}x}{\sqrt{x+\alpha}+\sqrt{x+\beta}} = 2\ln|\sqrt{x+\alpha} + \sqrt{x+\beta}| + c 
          \]
        \end{center}

      \subsubsection{Type 15: Inverse of sum of two $\frac{1}{n}\degree$ Variables}

        \begin{equation}
          \int \frac{\mathrm{d}x}{x^{\frac{1}{a}}+x^{\frac{1}{b}}} \ \ \ or \ \ \ 
          \int \frac{x^{\frac{1}{a}}\ \mathrm{d}x}{1+x^{\frac{1}{b}}}
        \end{equation}

        \begin{center}
          \textbf{Rule:} For $P = LCM(a,b)$, let $x^P = z$ and use other rules as necessary.
        \end{center}

        \textbf{Examples:}

        \begin{enumerate}
          \item \begin{flalign*}
              \int \frac{\mathrm{d}x}{x^{\frac{1}{2}}-x^{\frac{1}{4}}}
              & = \int \frac{4z^3 \mathrm{d}z}{z^2 - z} & let\ z^4 = x & \\
              & = \frac{4z^2 \mathrm{d}z}{z-1} & \implies \mathrm{d}x = 4z^3 \mathrm{d}z \\
              & = \int \frac{4z(z-1)+4(z-1) +4}{z-1}\ \mathrm{d}z \\
              & = 4\int z\mathrm{d}z + 4\int\mathrm{d}z + 4\int\frac{\mathrm{d}z}{z-1} \\
              & = 4\times\frac{z^2}{2} + 4z + 4\ln|z-1| + c \\
              & = 2x^\frac{1}{2} + 4x^{1}{4} + 4\ln|x^{\frac{1}{4}}-1| + c
            \end{flalign*}
          \item \begin{flalign*}
              & \int \frac{x^\frac{1}{2}\mathrm{d}x}{1+x^\frac{1}{3}} \\
              & = 6\int\frac{z^3 \times z^5\mathrm{d}z}{1+z^2} & let\ z^6 = x \implies 6z^5 \mathrm{d}z = \mathrm{d}x & \\
              & = 6\int\frac{z^8 \mathrm{d}z}{1+z^2} \\
              & = 6\int \frac{z^6(z^2+1)-z^4(z^2+1)+z^2(z^2+1)-(z^2+1)+1}{z^2+1}\ \mathrm{d}z \\
              & = 6\int z^6\mathrm{d}z - 6\int z^4\mathrm{d}z + 6\int z^2\mathrm{d}z - 6\int\mathrm{d}z + 6\int\frac{\mathrm{d}z}{z^2+1} \\
              & = \frac{6}{7}z^7 - \frac{6}{5}z^5 + 2z^3 - 6z + 6\arctan(z) + c \\
              & = \frac{6x^{\frac{7}{6}}}{7} - \frac{6x^{\frac{5}{6}}}{5} + 2x^{\frac{1}{2}} - 6x^{\frac{1}{6}}
                  + 6\arctan(x^{\frac{1}{6}}) + c
            \end{flalign*}
        \end{enumerate}
